\chapter{Conclusions and Future Work}
\label{ch:conclusions}

This dissertation has undertaken a comprehensive exploration into the operational inefficiencies in the collateral management systems within financial derivative trading. The primary focus has been on the development and evaluation of a blockchain-based solution, specifically on the BSV blockchain. The research has been conducted in alignment with industry standards set by the ISDA and has utilized sCrypt, a TypeScript-based Domain Specific Language, for the creation of smart contracts.

The study has made several key contributions. Firstly, a concrete methodology for tokenizing underlying collateral assets has been developed. This tokenization process allows for the efficient management and real-time valuation of assets, thereby potentially increasing market liquidity. Secondly, the research has outlined a robust digital twin representation for tokenized assets, demonstrating how these digital representations can encapsulate unique value per asset type. Thirdly, a hands-on Proof of Concept tokenizing crude oil as collateral has been presented, offering a tangible representation of how the proposed system might operate in practice. Lastly, the dissertation has examined the proposed solution from economic, financial, legal, and technological perspectives, providing a comprehensive understanding of its implications.

The research has shown that while the BSV blockchain is still in its early stages, it offers promising features such as microtransaction capability and scalability. However, there are technical hurdles in adapting existing ISDA frameworks to blockchain technology, which necessitate further research and development. On the regulatory and legal front, the study has found that the tokenization of collateral assets can be aligned with existing regulatory frameworks, although the transition would require a multi-stakeholder approach involving both financial and technological sectors. Economically and financially, the proposed system has the potential to reduce operational overheads, improve regulatory reporting, and streamline collateral allocation processes. However, the full economic benefits can only be realized through greater collaboration between the financial and tech sectors.

We provide the following suggestions for future work and alternative implementation based on the system hereby presented:

\begin{itemize}
    \item \textbf{Type System On-Chain}. As discussed on page \pageref{item:wallet}, implementing the CDM type system directly on the blockchain could address version mismatch issues by centralizing type definitions. This approach would enhance interoperability but would require a strategy for updating on-chain type definitions.

    \item \textbf{Intra-day Valuation Updates}. As discussed on page \pageref{ny3pm}, the current system updates the UTXO Set at a fixed time, 3pm ET, which may not capture market volatility. Future work could include a dynamic update mechanism that triggers when collateral valuation crosses a predefined threshold.

    \item \textbf{Updates of Legal and/or Economic Terms}. As discussed on page \pageref{item:legal_econ_terms}, while the current system focuses on valuation and margin-posting, future work could include the development of on-chain mechanisms for updating legal and economic terms in compliance with the CDM. This would enhance the system's adaptability to changing market conditions and regulatory requirements.

    \item \textbf{Additional Asset Types}. As discussed on page \pageref{item:other_assets}, future work could focus on implementing a comprehensive type system on the blockchain. This would allow for more granular control and validation of collateral types, extending beyond commodities to include complex financial products like interest rate swaps or indexes.
    
    \item \textbf{Oracles for Spot price and/or MTM}. As discussed on page \pageref{item:legal_econ_terms}, the initial design aimed to incorporate oracles for more accurate and consistent spot price and MTM data. Future work could focus on developing or integrating with emerging oracle services as the technical ecosystem matures, to mitigate the risks associated with valuation mismatches from traditional APIs.
    
    \item \textbf{FX Haircut}. As discussed on page \pageref{item:valuation}, the system currently relies on the WitnessOnChain oracle service for USDC-BSV exchange rate conversion. Future work could explore the integration of multiple oracles for redundancy and improved accuracy, as well as the development of a dynamic haircut model that adjusts in real-time based on exchange rate volatility.
    
    \item \textbf{On-chain Risk Calculations}. As discussed on page \pageref{item:risk_calc}, future work could explore the integration of more sophisticated risk models like VaR, SPAN, or Monte Carlo Simulations directly into the blockchain. This would address the issue of trust in off-chain calculations by making the parameters and results publicly auditable. Research could focus on optimizing these complex calculations to be more efficient on-chain or investigate hybrid models that perform part of the calculations on-chain and part off-chain while maintaining a high level of auditability. A possible starting point in this direction would be akin to zero-knowledge rollups in Ethereum. By leveraging zero-knowledge technology, computations are performed off-chain (thus not encountering the costly computational limits of performing operations on-chain), while inheriting the security and auditability of the blockchain by posting the proofs of the calculations on-chain \citep{rollups}. While the level of maturity of the Ethereum ecosystem in this regard is far more advanced than Bitcoin's, solutions built natively on BSV are starting to emerge, for example as described in \cite{rollups_bsv}. 

    \item \textbf{Tool to update CDM distributions in different languages}. As discussed on page \pageref{cdm_tool}, the existing Typescript implementation of the ISDA Common Domain Model (CDM) is outdated and requires manual reconciliation. Future work could focus on developing an automated tool that updates CDM distributions across multiple programming languages, ensuring they are in sync with the latest definitions and language-specific standards.

    \item \textbf{Fix for Generics}. The current system lacks support for generics in sCrypt, as detailed on page \pageref{item:generics}. Future work could focus on enhancing the sCrypt compiler to natively support generics, thereby maintaining type safety and reducing code duplication, or alternatively, developing a pre-compiler tool that automatically generates type-specific versions of generic interfaces.
    
    \item \textbf{Circular Dependencies}. Given the challenges posed by circular dependencies described on page \pageref{item:circular_deps}, future work could focus on a comprehensive refactoring of the CDM to eliminate these issues. This would not only improve type inference but also enhance the system's overall maintainability and compatibility with sCrypt.
    
\end{itemize}