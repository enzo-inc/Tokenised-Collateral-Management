\chapter{Introduction}
\label{ch:Introduction}
Derivatives \citep{quail2002financial}, as intricate financial instruments, rely on a myriad of interconnected systems—spanning finance, economics, law, and technology—for effective trading. The complex interplay between these systems often leads to conflicts that necessitate legal resolution or the cessation of trades to reconcile discrepancies in counterparties' information.

The 2008 financial crisis prompted international bodies to enforce stricter regulations to mitigate systemic risk \citep{ferran2012regulatory}. However, these expanded regulatory and legislative measures were imposed on an antiquated technological infrastructure not originally designed to accommodate them.

A challenging aspect of derivative trading is the management and payment of collateral. Collateral, assets temporarily transferred between parties, acts as a safeguard against counterparty risk during the derivative lifecycle. The current collateral administration process is fraught with issues. Movement of collateral assets often incurs considerable costs and operational overheads, sometimes necessitating the physical transfer of assets to third-party accounts. According to a report by the European Commission, the European Securities and Markets Authority (ESMA) has identified more than 400 trading data contributors, each currently reporting data in different manners \citep{eu_data_fragmentation}. The reason behind this is that current reporting
standards leave discretion in the interpretation of various reporting data fields, creating low quality market data reports and resulting in regulatory reporting
arbitrage, sometimes even paving the ground for deliberate mis-reporting of trades (the Financial Conduct Authority (FCA) constantly updates their list of reporting transaction fines \citep{fca_reporting_fines}, from whose frequency and magnitude we can infer the gravity of the issue). These inefficiencies frequently compel third parties to hold positions overnight and bridge unexpected holdings.

The International Swaps and Derivatives Association (ISDA) \citep{isda_home} has been working to standardize, digitize, and automate several crucial processes of derivative trading. A key focus has been the utilization of Smart Derivative Contracts to enhance infrastructure and mitigate challenges. By leveraging the transparency, accountability, and decentralization inherent in blockchain technology, Smart Derivative Contracts could automate many operational clauses of derivative agreements \citep{clack2018temporal}. This automation could reduce manual input (and thus error), cut reconciliation costs, improve regulatory reporting, and streamline the collateral allocation process. The end results could be diminished balance sheet risk and enhanced capital allocation for liquidity creation.

This dissertation proposes a tokenized collateral system implemented via smart contracts on the Bitcoin Satoshi Vision (BSV) blockchain \citep{bsv_home}. The aim is to refine the derivative trading lifecycle while adhering to existing regulatory, legislative, and business standards and processes. It evaluates the advantages of using the BSV blockchain to represent collateral, explores potential future directions for a settlement and payment layer trading such collateral, and analyzes the legal, financial, and economic implications of tokenized collateral. It also compares BSV with more widely-adopted distributed ledger systems like Ethereum \citep{dannen2017introducing} and Bitcoin \citep{nakamoto2008bitcoin}.

The research maps the collateral representation from the ISDA Common Domain Model (CDM) \citep{isda_cdm_factsheet} — the industry's most widely accepted standard—to one in sCrypt \citep{scrypt_home}, a domain-specific language (DSL) based on TypeScript \citep{typescript_home} for writing smart contracts on BSV. It examines the fractionalization of collateral value via satoshis and explores the use of off-chain oracles for providing real-world data streams to smart contracts.