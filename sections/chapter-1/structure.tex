\section{Structure of the Thesis}
\label{sec:structure}

The remainder of this thesis is structured as follows.

We provide the necessary background and technical preliminaries in \hyperref[ch:Background]{Chapter 2}. It is divided into several sections, starting with an overview of financial derivatives (Section \ref{sec:derivatives}). The life cycle of derivative trades (Section \ref{sec:derivative_trade_lifecycle}) and the specific challenges associated with collateral management are then elaborated (Sections \ref{subsec:collateral} and \ref{subsec:collateral_mgmt}). Further sections in Chapter 2 delve into existing industry and regulatory standards (Section \ref{sec:industry_approach}), such as the ISDA Common Domain Model and ISDA Create. Towards the end of the chapter, we also introduce the technical aspects of blockchain technology (Section \ref{sec:blockchain}), focusing on smart contracts (Section \ref{sec:smart_contracts}), Bitcoin Satoshi Vision (BSV) (Section \ref{subsec:bsv}), and oracles (Section \ref{subsec:oracles}). The chapter concludes with a regulatory and legal analysis of the status of tokenised assets (Section \ref{sec:asset_tokenisation}) and a literature review of current academic and industry advancements in the space (Section \ref{sec:lit_review}). 

\hyperref[ch:solutionArchitecture]{Chapter 3} presents the tokenised collateral system built on BSV. The chapter opens with an explanation of the key components that make up the system and discussion of the high-level architecture (Section \ref{sec:components}). This is followed by a comprehensive discussion on how ISDA's Common Domain Model has been translated and adapted to the constraints imposed by BSV and sCrypt, highlighting the interoperability and taxonomy challenges encountered (Section \ref{sec:mapping}). Finally, the flow of operations in a standard collateral re-evaluation process is described through a sequence diagram (Section \ref{sec:flow_of_events}).

\hyperref[ch:resultsDiscussion]{Chapter 4} brings forth the results and discussions based on the proposed architecture. This chapter is designed to provide a nuanced understanding of how the theoretical underpinnings translate to practical considerations. It discusses the role of custodians in this new architecture in Section \ref{sec:custodian}, as well as critically evaluating the benefits of the proposed solution from financial, economic, legal, and technological perspectives (Section \ref{sec:benefits}). Furthermore, it analyses the proposed tokenisation mechanism from a legal standpoint (Section \ref{sec:legal_considerations}). 

\hyperref[ch:conclusions]{Chapter 5} provides the conclusions and suggests avenues for future work. These suggestions encompass, among other things, enhancing the regularity of collateral re-evaluations, broadening the spectrum of supported asset categories, and integrating bespoke risk models into the computations. Ultimately, the importance of future collaboration between financial institutions such as ISDA and technology providers like sCrypt is highlighted as a means to advance the ecosystem in the future.