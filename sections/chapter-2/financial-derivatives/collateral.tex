\subsection{Collateral}
\label{subsec:collateral}
Collateral refers to the assets or pool of assets that are offered up by one party as a safety net, a guarantee against their contractual obligations \citep{isda_blueprint_collateral_processing}. Collateral can take many forms, ranging from cash and government securities to corporate bonds, equities, or any other asset deemed acceptable by the counterparties. The role of collateral is inherently protective, a safeguard designed to mitigate risk. Should a party default on their obligations, the collateral can be seized by the counterparty to counterbalance any financial loss resulting from the default.

The process of allocating collateral within derivative trades is multifaceted and dynamic. It's not simply a matter of assigning assets at the outset and leaving them be. Rather, it's a continuous recalibration, responding to shifts in the derivative position's market value, changes in the creditworthiness of the counterparties, and the constant ebb and flow of market volatility. This allocation process commences with the posting of an initial margin. This is essentially a percentage of the contract's total value, posted by both parties, acting as the first line of defense against potential market-induced losses.

In addition to the initial margin, counterparties are also required to post a variation margin \citep{IM}. Unlike the initial margin, the variation margin is not static. It's adjusted daily to accurately reflect the ever-changing market value of the derivative contract. Should market movements negatively impact a party's position, they may be called upon to post additional collateral, a process known as a margin call.

Moreover, it's worth noting that in certain circumstances, collateral can be subjected to reuse or rehypothecation. This essentially allows the receiving party to use the collateral for other purposes, such as pledging it for their own derivative trades. This practice, however, is subject to regulatory constraints and must be explicitly permitted in the collateral agreement.

In the world of derivative trades, the accurate valuation of collateral is paramount. From the moment it's posted, the value of collateral must be closely monitored and revalued regularly, often daily, to ensure it remains commensurate with the exposure. The quality of collateral also holds significant weight, with high-quality, highly liquid assets being the preferred choice because they can be swiftly sold if a counterparty default occurs.