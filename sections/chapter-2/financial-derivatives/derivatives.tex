\section{Financial Derivatives}
\label{sec:derivatives}

Financial derivatives are complex financial instruments whose value is contingent upon or derived from the value of another asset, referred to as the "underlying" asset \citep{quail2002financial}. These derivatives act as contracts between two or more parties, and their price fluctuates based on changes in the underlying asset, which can be virtually any item of value. Common examples of underlying assets include stocks, bonds, commodities (like gold, oil, or agricultural products), currencies, interest rates, and market indices.

Derivatives can take numerous forms, however they all fundamentally function as a way of shifting risk or opportunity between parties. Some of the most common types of derivatives include futures contracts, forward contracts, options, and swaps.

\begin{itemize}
    \item \textbf{Futures and Forwards}. These are agreements to buy or sell an asset at a specific future date at a pre-determined price. The key difference between the two is that futures are standardized contracts traded on an exchange, while forwards are privately traded over-the-counter (OTC) and can be customized to fit the needs of the parties involved.

    \item \textbf{Options}. These give the holder the right, but not the obligation, to buy (call option) or sell (put option) an asset at a specified price within a certain timeframe.

    \item \textbf{Swaps}. These are agreements to exchange one stream of cash flows for another. For example, in an interest rate swap, one party might agree to pay a fixed interest rate in exchange for receiving a variable rate from another party.
\end{itemize}

Derivatives play a crucial role in financial markets for three primary reasons:

\begin{itemize}
    \item \textbf{Risk Management and Hedging}. Companies and individuals alike use derivatives to reduce exposure to various risks. For instance, a manufacturing firm might use commodity futures to stabilize volatile input prices, or an international corporation might use currency swaps to mitigate the risk of exchange rate fluctuations. What sets derivatives apart in risk management is their ability to provide tailored solutions to hedge specific risks. Unlike traditional financial instruments, derivatives can be customized to match the exact duration, amount, and nature of the underlying exposure \citep{hammoudeh2013risk}. This precision allows entities to effectively neutralize their risk without over-hedging or under-hedging.

    \item \textbf{Speculation}. Traders and investors can use derivatives to profit from their predictions about changes in the price of the underlying asset. By correctly anticipating these price movements, they can potentially earn substantial returns. The leverage provided by derivatives is unparalleled. With a small initial margin or premium, investors can gain exposure to a much larger position in the underlying asset \citep{bartram2019corporate}. This means that even small movements in the price of the underlying can result in significant percentage returns (or losses) on the derivative position.

    \item \textbf{Arbitrage}. Derivatives can also be used to exploit price differences in different markets. Arbitrageurs aim to purchase an asset in one market and simultaneously sell it in another at a higher price, profiting from the price discrepancy. Derivatives offer a wider array of arbitrage opportunities due to their inherent complexity and the variety of contracts available. For instance, an arbitrageur can exploit mispricings between a stock and its futures contract, or between two different expiration dates of the same option. Additionally, derivatives often require less capital than purchasing the underlying asset directly, making arbitrage strategies more accessible \citep{panayides2006arbitrage}.
\end{itemize}