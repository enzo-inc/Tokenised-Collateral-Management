\section{Relevant Literature Review}
\label{sec:lit_review}
This literature review aims to provide a comprehensive assessment of the existing academic and professional literature on collateral management within the traditional derivatives market, the challenges and inefficiencies therein, and the advent of tokenized collateral in derivative trading. The chapter also explores the practicalities and legal considerations surrounding the deployment of tokenized collateral in financial markets.

Notably, \citep{anderson2014economics} argues that although there's abundant collateral supply, it is the infrastructural weaknesses that contribute to the immobilization of collateral, which can affect demand in another segment of the financial system. Complementing this perspective, \citep{jayeola2020inefficiencies} identifies the post-Global Financial Crisis inefficiencies in trade reporting, especially in over-the-counter (OTC) derivatives, positing that blockchain technology could serve as a viable solution to these inefficiencies. The potential for streamlining comes with a hefty price tag; according to \citep{Celent}, the financial industry would need to invest over \$53 billion in infrastructure and technology investments to upgrade and source new capabilities to achieve collateral efficiency and operational efficiency. Further complicating matters, the report also reveals that operational preparedness for derivatives clearing and collateralization remains a work in progress for nearly half of the firms surveyed. The increasing financial burden is also evident in the escalating amounts of Initial Margin (IM) and Variation Margin (VM) that need to be collected, which according to a 2022 ISDA report \citep{isda_key_trends}, reached \$1.4 trillion by the end of that year.

ISDA’s report on the cross-border fragmentation of global OTC derivatives \citep{isda_fragmentation} amplifies the complexity by pointing to a 77\% decline in volumes of cleared euro interest rate swaps between European and US dealers, highlighting the intricate challenges that arise as markets become more fragmented. These challenges are compounded by outdated methodologies; as Greenwich Associates suggests, the majority of firms still rely on manual methods for trade confirmation and reconciliation, a practice that not only hampers speed but also consumes 60\% of the budget for cleared derivatives processing in North America. As a transformative step, \citep{feenan2021decentralized} propose a paradigm shift towards decentralized Financial Market Infrastructures (dFMIs), even though they acknowledge the associated complexities of complete decentralization.

Contrary to the shortcomings of the traditional system, tokenized collateral in derivative trading appears promising. \citep{morini2016blockchain} furnishes a strong business case and a more formal structure for Smart Collateral Contracts, positing that tokenized collateral could be aligned seamlessly with traditional risk management strategies and CCPs. \citep{schar2021decentralized} goes even further by signaling the strong potential for tokenized collateral to fundamentally alter derivatives trading through decentralized finance and on-chain asset management. Although \citep{lesavre2020blockchain} provides a generalized overview of asset tokenization, they do not specifically delve into its financial applications. Yet, the promise of tokenized collateral is also countered by legal and regulatory impediments. For instance, \citep{giancaspro2017smart} and \citep{raskin2016law} question the legal enforceability of smart contracts, probing the delicate intersection of law and smart contract coding. Similarly, \citep{ferreira2020emerging} talks about the regulatory landscape that is yet to fully embrace blockchain-based tokens, pointing to the potential hurdles that might arise as regulators begin to closely scrutinize this growing space.

The existing body of literature, while comprehensive in assessing the inefficiencies and challenges of traditional collateral management systems and the prospects of tokenized collateral, leaves a critical gap in terms of implementing these innovations on blockchain platforms, specifically on the BSV network. This oversight is particularly significant given BSV's unique features and capabilities that may offer distinct advantages for collateral management. The literature falls short in providing a detailed roadmap for actualizing tokenized collateral systems on BSV, and there is limited focus on the specific operational, legal, and regulatory considerations that would be unique to this particular blockchain. The present work aims to bridge this gap by not only conducting an in-depth analysis of how tokenized collateral management could be effectively implemented on BSV but also by addressing the idiosyncratic challenges and opportunities that this blockchain platform presents. In doing so, the research contributes a nuanced perspective that could serve as a blueprint for future implementations and regulatory discussions, thereby filling a significant void in the current academic and professional discourse.