\subsection{Oracles}
\label{subsec:oracles}
Oracles, within the context of blockchain and smart contract technologies, serve as intermediaries that provide the bridge between blockchain systems and the external world \citep{al2020trustworthy}. They relay real-world data to smart contracts on the blockchain, enabling these decentralized applications to interact and integrate with off-chain data and systems. Given that blockchains are deterministic systems, they can't directly interact with the external world because they operate based on pre-defined rules and consensus mechanisms that ensure security and immutability within the network. Since blockchains are designed to be isolated systems to maintain their integrity and trustworthiness, any interaction with external data sources poses the risk of introducing vulnerabilities or inaccuracies. Thus, they need a trusted channel to provide them the necessary data, and that's where oracles come in.

Oracles can be of various types, such as input oracles, output oracles, cross-chain oracles, and compute-enabled oracles \citep{types_oracles}. Input oracles fetch real-world data and deliver it to the blockchain for smart contracts to use. Output oracles, on the other hand, enable smart contracts to dispatch instructions to external systems, prompting them to carry out specific actions. Cross-chain oracles facilitate data and asset movement between different blockchains, thus enabling interoperability. Compute-enabled oracles provide off-chain computation capabilities that are impractical to perform on-chain due to various constraints.

Oracles can significantly enhance the efficiency and robustness of collateral management systems in several ways:

\begin{itemize}
    \item \textbf{Real-time Price Feeds} \citep{Decentralized_Data_Feeds}. The foundation of collateral management is the accurate valuation of collateral assets. Oracles, especially input oracles, deliver real-time market data relating to the prices of assets that serve as collateral in smart contracts. As an example, in scenarios where a specific cryptocurrency or token is put forth as collateral, an oracle can supply its latest market price, ensuring that the valuation of the collateral is as precise as possible. In the context of decentralized finance (DeFi), such data streams can stimulate automated reactions when prices hit predetermined thresholds - for instance, it may launch a margin call when the collateral's value dips below a certain level, thus ensuring prompt action and prevention of potential risks.

    \item \textbf{Risk Management} \citep{volatility_oracles}. Proficient collateral management hinges on robust risk management protocols, and oracles play an instrumental role in this area. They have the capacity to supply data concerning the volatility and liquidity of assets under consideration as collateral. This information aids in assessing associated risks - for example, an asset with high volatility brings with it a higher risk of a sharp decline in its value, which could result in the collateral's value being insufficient. Oracles deliver real-time data feeds that can support these risk assessments and guide better informed, proactive decision-making.
    
    \item \textbf{Automation of Collateral Calls}. Oracles and smart contracts together contribute to automating collateral calls, each serving distinct roles within the system. In this setup, oracles function as real-time data feeds that continuously monitor market conditions and the valuation of the counterparty's collateral. When these oracles detect that the market circumstances have changed unfavorably, thereby devaluing the collateral to an insufficient level, they pass this critical information to the smart contract. On the other hand, the smart contract holds the pre-defined logic for executing a collateral call based on the information received from the oracle. It determines if the collateral devaluation warrants an automatic call, calculates the additional amount of collateral required, and then triggers the collateral call to the counterparty. The smart contract also facilitates the secure, trustless transaction of posting additional collateral, making the entire process more efficient and transparent. This is the use case the we propose as part of system described in Chapter \ref{ch:solutionArchitecture}.

 
    \item \textbf{Regulatory and Compliance Checks}. Oracles can play a critical role in not just retrieving data related to regulatory obligations and compliance checks, but also in ensuring that smart contracts remain up-to-date with real-time changes in regulations and industry-standard practices. This is especially significant given that regulations and compliance standards are often subject to change, and smart contracts, once deployed, are immutable by nature \citep{zheng2020overview}. To address this, smart contracts must be designed with "upgradability" in mind, allowing for the modification of their logic without requiring the redeployment of the entire contract. Various techniques exist to achieve upgradability, such as using external libraries, data separation, or the proxy pattern \citep{meisami2023comprehensive}. In particular, the proxy pattern \citep{proxy_pattern} is a technique wherein the main smart contract delegates its logic to another contract (often termed the "implementation contract"). The proxy contains the state variables and remains consistent, while the logic and data manipulation are conducted in the implementation contract. When regulations change, a new implementation contract can be deployed and linked to the existing proxy, ensuring that compliance-related data is always current and that collateral evaluation criteria align with the latest regulatory requirements. At the time of writing and to the best of our knowledge, no specific  oracles solely dedicated to providing information about standards and regulations exist. However we envision, and in fact suggest, ISDA launching its own oracle service to provide requirements about, for example, the eligibility of different types of collateral \citep{cdm_website}. 
    
    \item \textbf{Cross-chain Interoperability} \citep{crosschain_oracles}. In a context where multiple blockchains are in operation, the same asset could exist on several chains. Cross-chain oracles in such scenarios can be invaluable, enabling smart contracts to identify and engage with these assets across different chains. This significantly improves the flexibility and scope of the collateral management process.
    
    \item \textbf{Settlement and Reconciliation}. Oracles can also expedite the settlement and reconciliation process inherent to collateral management. Output oracles, for instance, can engage with traditional banking systems to initiate payments based on the stipulations of a smart contract, thereby seamlessly aligning traditional and decentralized finance systems \citep{tradfi_defi_oracles}.
\end{itemize}