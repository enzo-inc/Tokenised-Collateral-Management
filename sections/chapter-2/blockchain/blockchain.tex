\section{Blockchain}
\label{sec:blockchain}
Blockchain is a decentralized and distributed ledger technology that allows multiple participants to maintain a shared and tamper-evident record of transactions or information in a secure and transparent manner \citep{morini2016blockchain}. It was originally introduced as the underlying technology for cryptocurrencies like Bitcoin, but its potential applications have expanded beyond digital currencies.

At its core, a blockchain is a chronological chain of blocks, where each block contains a batch of validated transactions or data. Each block is linked to the previous one through a cryptographic hash, creating a chain of information that is extremely difficult to alter. Throughout the rest of this work we use the term "immutable" to describe this chain, however it's important to note that blockchain blocks can technically be changed. However, any alteration to a previous block would become immediately evident when the cryptographic hashes are checked, requiring the modification of all subsequent blocks. This makes the blockchain highly resistant to tampering and ensures the integrity of the data.

Blockchain is particularly suited to address the challenges encountered in the collateral management lifecycle described in section \label{sec: infratrcture_challenges}. In particular,

\begin{itemize}
    \item \textbf{Asset Selection}. By incorporating blockchain technology, the process of asset selection can be significantly standardised. This is achievable by establishing a decentralised protocol for defining eligible collateral, thereby eliminating the discretion and free-form nature of the selection process. This protocol could also include a consensus mechanism to agree on definitions for asset types such as HQLAs, reducing misunderstandings and potential disputes.

    \item \textbf{Margin and Interest Calculation}. Blockchain's smart contracts, which are programmable contracts that automatically execute when certain conditions are met, could automate the manual, error-prone process of margin and interest calculation. By setting the terms of agreements, including rates and day count, as the conditions in the smart contract, the calculations would be automatically performed and agreed upon by all parties involved, thereby eliminating discrepancies and reducing settlement risks.
    
    \item \textbf{Trade Transaction Management}. Blockchain's immutability and real-time transaction records can significantly enhance trade transaction management. By storing every new trade and amendment on the blockchain, all parties can monitor the portfolio volatility and address trade-matching issues immediately, rather than waiting for the following day. This would minimise mismatches and unmatched trades, reducing the disputes in the margin and collateral process.
    
    \item \textbf{Record Keeping and Reconciliation}. With blockchain technology, record keeping and reconciliation could be streamlined and automated. Each transaction and adjustment made during the collateral lifecycle would be recorded in real-time on the blockchain. This permanent, transparent ledger would provide a single source of truth for all participants, thus eliminating data fragmentation, reducing the inaccuracies in trade reporting, and cutting down the time and cost of reconciliation.
    
    \item \textbf{Operational Challenges}. The use of blockchain could also mitigate the operational challenges involved in the collateral lifecycle. The movement of assets used as collateral could be represented as token transfers within the blockchain, bypassing the need for physical transfer of assets. This could expedite the process, reduce costs, and make it easier to comply with regulatory rules. By tokenizing assets, principals could also easily release assets as collateral, enhancing operational efficiency. The benefits of efficient release and re-allocation of collateral are described in detail in Section \ref{subsec:benefits_financial}.
\end{itemize}