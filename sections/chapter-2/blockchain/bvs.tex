\subsection{Bitcoin Satoshi Vision}
\label{subsec:bsv}
Bitcoin Satoshi Vision (BSV) is a cryptocurrency and blockchain network that emerged from a hard fork split from Bitcoin Cash (BCH) in November 2018 \citep{bsv_home}. It was created in response to disagreements within the Bitcoin Cash community over proposed changes to the protocol. BSV proponents, led by Craig Wright and Calvin Ayre, sought to adhere more closely to what they consider to be the original vision of Bitcoin's anonymous creator, Satoshi Nakamoto. This vision encompasses principles of larger scalability, minimal transaction fees, and robust data handling, making BSV uniquely suited to address the challenges in the collateral management lifecycle. In particular, the benefits offered by BSV in the collateral management process become even more evident when compared to alternative distributed ledger platforms such as Ethereum Bitcoin:

\begin{itemize}
    \item \label{item:scalability} \textbf{Scalability}. BSV's main advantage is its scalability, which is primarily achieved by significantly increasing the block size limit as compared to Bitcoin and Ethereum. The original Bitcoin and Ethereum networks have opted for smaller block sizes (1 MB, fixed-size blocks for Bitcoin \citep{btc_block_size_limit} and variable-sized blocks for Ethereum --- up to a ceiling of 30 million "gas" units \citep{eth_block_size}, where "gas" is the unit of measurement for computational effort that is required to perform various operations, such as executing smart contracts, making transactions, or storing data \citep{eth_gas}) to maintain decentralization and security. Smaller block sizes are easier for individual nodes to process, thereby encouraging more participants and maintaining the decentralized nature of the blockchain. However, the trade-off is a limited throughput of transactions, which can slow down the network and increase transaction fees during periods of high demand \citep{singh2020computing}. BSV, on the other hand, has decided to prioritize scalability by allowing much larger block sizes (up to 4GB at the time of writing \citep{bsv_block_size}), aiming to accommodate more transactions per block. This enables quicker processing of transactions, creating a responsive system that is ideal for managing daily portfolio volatility and facilitating trade transactions. Fixed block sizes are generally used as a compromise between scalability and network health; variable block sizes could lead to inconsistencies in block propagation and validation times, potentially compromising the security and integrity of the network \citep{singh2020computing}.

    \item \textbf{Data Handling}. BSV provides robust data handling capabilities, allowing for comprehensive transaction data to be stored directly on the chain. While Ethereum also supports data storage, its higher transaction fees for complex data operations may limit efficiency. BSV's capabilities ensure an efficient, transparent, and immutable record of collateral lifecycle events, which dramatically improves record keeping, reconciliation, and reporting accuracy.
    
    \item \label{item:smart_contracts} \textbf{Smart Contracts}. BSV's smart contracts offer an economical approach to automating processes in the collateral management lifecycle. Unlike Ethereum or Bitcoin's smart contracts, which can incur high gas fees, BSV has significantly lower transaction costs, though it is not entirely without fees --- at the time of writing (5th September 2023), the average transaction cost is \$0.81 on Ethereum \citep{eth_tx_fee}, \$1.40 on Bitcoin \citep{btc_tx_fee} and \$0.00000367 on BSV \citep{bsv_tx_fee}. This makes it more cost-effective for executing complex automation tasks, such as margin and interest calculations and asset selection. Additionally, BSV, similarly to Bitcoin, incorporates specific safeguards such as opcode limits and script size restrictions to minimize the risk of "runaway processes" like infinite loops \citep{btc_limits_safeguards}. This contrasts with Ethereum, where the presence of high gas fees often acts as a de facto limitation against such inefficiencies but doesn't provide built-in safeguards in the protocol \citep{eth_loops}. BSV's approach ensures a more streamlined and reliable experience for executing smart contracts, without relying solely on transaction costs as a deterrent against poorly optimized code.
    
    \item \textbf{Micropayments}. BSV's low transaction fees make it highly effective for micropayments, allowing for flexible interest payment schedules and streamlined settlements. This directly contrasts with Ethereum, where high gas fees make small transactions economically inefficient. BSV's affordability in this context significantly minimizes the need for manual reconciliation processes, such as matching transactions, verifying records, and resolving discrepancies, thereby reducing both time and labor costs.
    
    \item \textbf{Tokenization}. Both BSV and Ethereum platforms offer the capability to tokenize assets, thereby streamlining the collateral management lifecycle. Tokenization on these platforms eliminates the need for physical asset transfer, enhancing operational efficiency. However, the two differ in terms of transaction costs. BSV offers a more cost-effective solution for asset tokenization and transfer, while Ethereum, although equally functional in allowing tokenization, comes with higher transaction fees \citep{di2020tokens}. This makes BSV a more economical choice for businesses looking to manage operational challenges in collateral management without sacrificing functionality.
\end{itemize}