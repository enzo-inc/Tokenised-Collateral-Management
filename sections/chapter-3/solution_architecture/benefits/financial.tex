\subsection{Financial}
\label{subsec:benefits_financial}

\begin{itemize}
    \item \textbf{Reduction in operational overheads}. The operational overheads and costs associated with derivatives trading in the realm of collateral management can be substantial. The financial services sector could save more than €4 billion annually in collateral management costs by addressing operational inefficiencies \citep{accenture}. These costs stem from the need for specialized software, manpower, and compliance measures to manage the complex collateral requirements associated with derivatives contracts. Additionally, the advent of regulations like Dodd-Frank in the U.S. \citep{dodd-frank} and EMIR \citep{EMIR} in Europe has led to increased reporting and margin requirements, further escalating costs \citep{costs_doddfrank_emir}. Firms also incur opportunity costs by tying up capital as collateral that could otherwise be invested. A report by Deloitte estimates that collateral requirements could tie up as much as \$1.9 trillion in high-quality assets \citep{deloitte_costs}. Therefore, the operational overheads in collateral management are not just a cost center but also a strategic concern that impacts liquidity and capital efficiency. Our solution promises to relieve trading entities of the operational overheads by proposing a standardized format to tokenize collateral and automating the movement of assets.


    \item \textbf{Enhanced liquidity}. Enhanced liquidity in derivatives trading, particularly through the ease of trading and transferring assets, has a profound impact on the market. By making previously illiquid assets more accessible, the market experiences a boost in trading volume and efficiency. For instance, the securitization of illiquid assets like mortgages or loans into tradable derivatives allows a broader range of investors to participate in markets that were previously inaccessible. According to statistics reported by the Bank for International Settlements (BIS), the global derivatives market has grown to an estimated \$640 trillion in notional amount outstanding, partly fueled by the increased liquidity of assets \citep{bis_costs}. The advent of digital assets has further augmented this trend. Cryptocurrencies and tokenized assets, being easily tradable on various digital platforms, have introduced a new layer of liquidity. They have enabled even more participants to engage in the market, making assets like real estate or art, once considered highly illiquid, more accessible through tokenization \citep{digital_asset_tokenization}. This enhanced liquidity not only fosters market stability by allowing for more seamless price discovery but also promotes economic growth by facilitating capital allocation. Furthermore, it reduces the cost of capital for issuers and provides investors with diversified investment opportunities.

    While the benefits of enhanced liquidity through asset tokenization and digital trading platforms are significant, it's crucial to consider the complexities involved. For instance, the mere act of making a traditionally illiquid asset more tradable does not automatically translate to increased market liquidity. The Central Limit Order Book (CLOB) systems, commonly used in asset trading, may not be well-suited for illiquid assets, as Market Makers often require a wide spread to mitigate the risks associated with the asset's volatility \citep{maureen2001overview}. To address these challenges, we propose two alternative approaches to consider. One such strategy is to restrict the trading window for illiquid assets, thereby concentrating market activity and potentially stabilizing prices. Another approach could be the adoption of Automated Market Maker (AMM) systems, a concept borrowed from decentralized finance (DeFi). AMMs operate on a liquidity pool model where participants deposit assets into a smart contract. These pools can then facilitate trades directly between buyers and sellers without the need for a traditional Market Maker. In the context of illiquid assets, an AMM system allows participants to provide liquidity directly, making the market more robust and efficient \citep{aoyagi2020liquidity}. These nuanced strategies aim to foster a more stable and liquid market, facilitating better price discovery and capital allocation.
\end{itemize}