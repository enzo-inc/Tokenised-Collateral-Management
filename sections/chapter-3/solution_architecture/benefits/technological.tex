\subsection{Technological}
\label{subsec:benefits_technological}

\begin{itemize}
    \item \textbf{Scalability}. The choice of BSV as the underlying blockchain technology for the proposed collateral management system is not arbitrary but is informed by a set of unique technological advantages that it offers. First and foremost is the issue of scalability. BSV is designed to handle a high transaction throughput, a critical requirement for the fast-paced, high-volume nature of derivative trading. Unlike other blockchains that struggle with scalability issues, BSV can handle larger block sizes, thereby facilitating more transactions per second \citep{zohar2015bitcoin}. To provide a quantitative perspective, BSV can process up to 2,000 transactions per second (tps) \citep{tartan2021scalable}, compared to Bitcoin's 7 tps \citep{gobel2017increased} and Ethereum's 30 tps \citep{li2020comparative}. This increased throughput is facilitated by BSV's capacity for larger block sizes—up to 2GB as opposed to Bitcoin's 1MB and Ethereum's variable, but smaller, block size --- as discussed on page \pageref{item:scalability}. The larger block size not only allows for more transactions per block but also reduces the likelihood of transaction backlog, ensuring faster processing times. In terms of overhead, BSV offers lower transaction fees, with average fees orders of magnitude smaller than Ethereum's or Bitcoin's, as mentioned on page \pageref{item:smart_contracts}.

    \item \textbf{Security}. Security is another cornerstone. BSV offers a robust security protocol that can withstand various types of attacks, making it a reliable platform for managing financial assets. Its proof-of-work consensus algorithm and the cryptographic techniques employed ensure the integrity and immutability of data \citep{nakamoto2008bitcoin}.

    \item \textbf{Interoperability}. The system is interoperable with other financial systems. This is particularly important for derivative trading entities that operate across different blockchain ecosystems. For instance, a functionally-equivalent representation of an ERC-20 fungible token from the Ethereum blockchain \citep{erc_20_eth} could be implemented on BSV \citep{erc_20_bsv}, thereby broadening utility and adoption of both ecosystems. Details on how ERC-20 tokens could be implemented on BSV are outiside the scope of this project and are left as future work.

\end{itemize}