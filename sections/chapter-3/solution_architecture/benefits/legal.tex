\subsection{Legal}
\label{subsec:benefits_legal}

\begin{itemize}
    \item \textbf{Regulatory compliance}. In the traditional system of collateral management for derivatives trading, regulatory compliance has often been a cumbersome process, fraught with inefficiencies such as manual record-keeping and auditing. The cost of compliance is substantial; according to a 2020 report by Thomson Reuters, financial firms spend approximately \$180 billion annually on compliance and regulatory obligations \citep{Thomson_Reuters}. The immutability of the blockchain ensures that once a transaction is recorded, it cannot be altered or deleted. This serves as a robust mechanism for audit trails, aiding in compliance with regulations that require firms to maintain historical data for several years. For example, the Dodd-Frank Act \citep{dodd-frank} requires swap dealers to keep records for as long as the swap is active and for five years thereafter \citep{SEC}. The blockchain's immutable nature inherently satisfies this requirement. Additionally, smart contracts can be programmed to automatically enforce compliance rules, such as minimum collateral requirements or maximum leverage ratios. This reduces the manual effort involved in ensuring compliance and minimizes the risk of human errors.

    \item \textbf{Dispute resolution}. The absence of a single, transparent source of truth in traditional systems often leads to discrepancies in collateral valuation, margin calls, and other contractual obligations, thereby causing disputes that are costly and time-consuming to resolve. The BSV blockchain serves as an immutable ledger, recording all transactions and collateral adjustments, thus eliminating the possibility of data manipulation and reducing the scope for such disputes. This feature is particularly relevant given the recent remarks by ISDA on the necessity to harmonize data reporting rules across jurisdictions and ensuring consistent data sets for regulators \citep{isda_transparency}.
    

    Moreover, the traditional dispute resolution process is bogged down by bureaucratic inefficiencies and delays, exacerbated by the involvement of multiple parties with disparate record-keeping systems. Smart contracts on the BSV blockchain can be programmed to automatically execute actions like margin calls based on predefined conditions. This automation minimizes human errors and ensures compliance with the terms of derivative contracts, thereby reducing the likelihood of disputes arising from non-compliance. Such automation aligns with the recommendations made in DTCC's "The Changing Face of Derivative Reporting" to improve efficiency \citep{dtcc_reporting}.
\end{itemize}