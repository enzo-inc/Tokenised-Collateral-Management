\subsection{Economic}
\label{subsec:benefits_economic}

\begin{itemize}
    \item \textbf{Increased market efficiency}. Traditionally, the settlement of trades often operates on a "T+2" basis, meaning that the transaction is finalized two business days after the trade is executed. This delay introduces a range of inefficiencies and risks, including counterparty risk and the need for more extensive collateral management \citep{t_2_settlements}. Our solution promises to revolutionize this paradigm by enabling "T+0" settlements—same-day settlement of collateral operations. The acceleration to "T+0" is not merely a technological feat but a transformative shift towards increased market efficiency. According to a report by McKinsey \citep{mckisney_tokenization}, shorter settlement times generate significant savings in high-interest-rate environments such as at the time of writing \citep{BoE_rates}. For investors, these savings may be the greatest near-term impact and the main reason why the business case for tokenization is specifically now ripe for delivering advantages.
    

    \item \textbf{Risk mitigation}. In addressing the inefficiencies and vulnerabilities in collateral management, our solution offers substantial improvements in risk mitigation. Firstly, trust between parties is enhanced; the transparency and immutability of blockchain transactions eliminate the need for intermediaries, thereby reducing information asymmetry and counterparty risk, as highlighted by Deutsche Bundesbank \citep{Deutsche_Bundesbank}. Operational risks — such as the risk of delayed settlements or human errors  — are also mitigated as the BSV'S low latency enables real-time settlement, thus increasing liquidity. A study by the Depository Trust \& Clearing Corporation (DTCC) found that 30\% of trades have discrepancies due to manual reporting and human errors \citep{DTCC}. Lastly, the cryptographic security inherent in blockchain technology mitigates the risk of fraud and unauthorized transactions.
\end{itemize}