\section{Legal Consideration of Tokenised Collateral}
\label{sec:legal_considerations}
In Section \ref{sec:asset_tokenisation}, we delineate between various tokenisation models. We now classify the system proposed in this study as a hybrid model that combines features from both the Registered and Claims paradigms. From the Registered model, our system adopts an identity layer, thereby anchoring the ownership of assets to verified identities maintained in a regulated registry. This element is pivotal for adhering to the rigorous legal and regulatory compliance standards often encountered in collateral agreements. Conversely, the system incorporates elements from the Claims model by enabling participants—whether they are the counterparties or custodians acting on their behalf—to engage in direct interactions through the utilization of operator-deployed smart contracts. Additionally, it is important to note that within the proposed system, the tokenised collateral does not constitute an asset per se. Rather, it serves as a digital equivalent, the ownership and control rights of which are established through contractual agreements.

To ensure legal and regulatory compliance, the operator of the proposed system must adhere to a set of key requirements, outlined here in a non-exhaustive list. These include Anti-Money Laundering (AML) regulations such as the United States' Bank Secrecy Act (BSA) \citep{BSA} and the United Kingdom's Money Laundering Regulations 2017 \citep{uk_aml}. Furthermore, Know Your Customer (KYC) procedures are obligatory under these frameworks to verify the identities recorded in the registry. Smart contracts deployed by the operator must also comply with laws governing electronic signatures, such as the Electronic Signatures in Global and National Commerce Act (E-SIGN Act) \citep{e-SIGN} in the U.S. or the EU's eIDAS regulation \citep{eIDAS} for electronic identification and trust services. In terms of establishing control and ownership rights over digital and tokenized assets, the Uniform Commercial Code (UCC) Article 9 \citep{ucc_9} serves as a relevant legal framework.