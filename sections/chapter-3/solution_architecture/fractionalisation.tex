\section{Fractionalisation of Value}
\label{sec:fractionalisation}
From my reading so far, fractionalisation of value is useful mostly for retail investors who would have access to asset classes and risks that would otherwise lie beyond their capacity. Fractionalisation would lead to more inclusive access of small and retail investors, while enabling global pools of capital to reach parts of the financial markets previously reserved to large investors. But derivatives are usually reserved for insitutional investors as they are very complex and risky, what is the point of fractionalisation for these investors?

In the context of collateral, fractionalisation would help with using traditionally illiquid assets as collateral.

From \href{https://www.icmagroup.org/market-practice-and-regulatory-policy/repo-and-collateral-markets/icma-ercc-publications/frequently-asked-questions-on-repo/6-what-types-of-asset-are-used-as-collateral-in-the-repo-market/}{here} it looks like domestic government bonds form 90\% of the collateral used in the repo market. Repo using collateral other than high-quality government bonds is often called credit repo. Private sector assets form the smallest sector of the repo market (corporate bonds, equity, covered bonds, MBSs, ABSs, Money Market securities, bank loans, gold). Could fractionalisation make it easier to user this type of collateral??